\begin{preamble}

\iffinal{}{\newpage}

\begin{DUTdedications}
\begin{center}

I dedicate this thesis to my family, and to my wife,\\
who supported me unconditionally throughout my \\
career as a scientist.\\

\end{center}
\end{DUTdedications}

\iffinal{}{\newpage}

\begin{acknowledgments}
I have many people to thank for making this work a
possibility. Firstly, I would like to thank my advisor, Dr. David
Goldberg. His guidance, support, and most of all his patience provided
the framework for which I was able to build my work from. I would also
like to thank my dissertation committee members, Dr. Michael Vogeley,
Dr. Gordon Richards, Dr. Luis Cruz, and Dr. Andrew Hicks, for their
constructive criticism and support of my research.

I would also like to acknowledge fellow graduate students for their
assistance and support. A special thanks to Justin Bird, Markus
Rexroth, Frank Jones, Crystal Moorman, and Vishal Kasliwal for
allowing me to bounce ideas off of them, which was a truly important
but time consuming process.
\end{acknowledgments}

\iffinal{}{\newpage}

\tableofcontents 
\iffinal{}{\newpage}

\listoftables
\iffinal{}{\newpage}

\listoffigures 
\iffinal{}{\newpage}

\begin{abstract}

\ifdaring{\setstretch{1.3}}{\setstretch{1.6}}

We explore the intrinsic distribution of dark matter within galaxy
clusters, by combining insights from the largest {\em N}-body
simulations as well as the largest observational dataset of its kind.
  
  Firstly, we study the intrinsic shape and alignment of isodensities of galaxy
  cluster halos extracted from the MultiDark MDR1 cosmological
  simulation. We find that the simulated halos are extremely prolate
  on small scales and increasingly spherical on larger ones. Due to
  this trend, analytical projection along the line of sight produces an
  overestimate of the concentration index as a decreasing function of
  radius, which we quantify by using both the intrinsic distribution
  of 3D concentrations ($c_{200}$) and isodensity shape on weak and
  strong lensing scales. We find this difference to be $\sim 18\%$
  ($\sim 9\%$) for low (medium) mass cluster halos with intrinsically
  low concentrations ($c_{200}=1-3$), while we find virtually no
  difference for halos with intrinsically high
  concentrations. Isodensities are found to be fairly well-aligned
  throughout the entirety of the radial scale of each halo 
  population. However, major axes of individual halos have 
  been found to deviate by as much as $\sim 30^{\circ}$. We also
  present a value-added catalog of our analysis results, which we have
  made publicly available to download. 

  Following that, we then turn to observational measurements galaxy clusters.
Scaling relations of clusters have made them
particularly important cosmological probes of structure formation. In this
work, we present a comprehensive study of the relation between
two profile observables,  concentration ($\mathrm{c_{vir}}$) and mass
($\mathrm{M_{vir}}$). We have collected the largest known sample of measurements from
the literature which make use of one or more of the following reconstruction
techniques: Weak gravitational lensing (WL), strong gravitational lensing (SL),
Weak+Strong Lensing (WL+SL), the Caustic Method (CM), Line-of-sight Velocity
Dispersion (LOSVD), and X-ray. We find that the concentration-mass (c-M)
relation is highly variable depending upon the reconstruction technique
used. We also find concentrations derived from dark matter only simulations
(at approximately $\mathrm{M_{vir} \sim 10^{14} M_{\odot}}$) to be
inconsistent with the WL and WL+SL relations at the $\mathrm{1\sigma}$
level, even after the projection of triaxial halos is taken into
account. However, to fully determine consistency between simulations and
observations, a volume-limited sample of clusters is required, as
selection effects become increasingly more important in answering
this. Interestingly, we also find evidence for a steeper WL+SL relation as
compared to WL alone, a result which could perhaps be caused by the varying
shape of cluster isodensities, though most likely reflects differences in
selection effects caused by these two techniques. Lastly, we compare
concentration and mass measurements of individual clusters made using
more than one technique, highlighting the magnitude of the potential bias
which could exist in such observational samples.

Finally, we explore the large-scale environment around galaxy clusters using
spectroscopically confirmed galaxies from the Sloan Digital Sky Survey
(SDSS) Data Release 10. We correlate the angular structure of the distribution
of galaxies (out to a distance of $\mathrm{10 h^{-1}\, Mpc}$) around 92 galaxy
clusters with their corresponding mass and concentration measurements. We find
that the orientation of the cluster environment on this scale has little impact
on the value of cluster measurements. 

\end{abstract}

\iffinal{}{\newpage}

\end{preamble}