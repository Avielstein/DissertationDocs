\chapter{Conclusions}

The internal distribution of mass within galaxy clusters is a crucial test of
physics on the largest scales in the universe. Maps of clusters can be made in
numerous ways, however, direct comparison becomes hard to make for a few
reasons. Methods which depend upon galaxies to trace the mass 
(e.g. - the caustic method, and line-of-sight velocity dispersion),
gravitational lensing (weak, strong, and weak+strong), and methods which relate
the hot X-ray emitting gas to the total mass of the cluster, each come with
their own set of assumptions, and probe varying masses, orientations, and
concentrations from the cosmic population. 

Furthermore, the relationship between the cluster concentration and mass
(profile properties which the aforementioned techniques aim to measure),
has been observed to be significantly different between large-scale,
dissipationless {\em N-}body simulations and observations. This problem has been 
dubbed, ``The over-concentration problem'', and is the primary motivation for
the work we have done.

By studying galaxy clusters forming in the MultiDark MDR1 cosmological
simulation (Chapter 2), we are able to observe their intrinsic 3-dimensional
density profiles. This vantage point is not currently available
observationally. We find that dynamically relaxed clusters are generally 
well-fit by the universal NFW profile, with co-aligned isodensities. However,
upon projection along the line-of-sight direction, we find that the concentration
becomes an aperture-based quantity, caused by one very important feature of
clusters. Cluster isodensities, while remaining largely coaxial with one
another, are not the same shape on all radial scales throughout the halo. This
difference in shape can cause an upward bias of $\mathrm{\sim20\%}$ in
2-dimensional concentrations on strong lensing scales ($\mathrm{\sim 0.5\cdot
  r_{200}}$) as compared to weak lensing scales ($\mathrm{\sim r_{200}}$), when
halos are viewed along their major axes. This rise in galaxy cluster axis
ratios is something studies do not currently account for, and has the potential
to cause systematic biases within the concentration-mass relations generated by
different observational reconstruction techniques.

Motivated by this prediction, we then look for this novel characteristic difference in the
concentration-mass relations of a sample of observational measurements in
Chapter 3. We collect all known concentration and mass measurements from the
literature, for which we quantify the c-M relation for each of six common
reconstruction techniques (X-ray, WL, SL, WL+SL, CM, and LOSVD; see section
3.2). Though we cannot constrain the relation for strong lensing clusters (due
to a lack of data), we do indeed find a somewhat steeper slope for WL+SL than
we do for WL alone. However, observational relations are subject to an
additional potential bias as compared to ones generated by
simulations. Specific instruments or reconstruction techniques carry with them
observational thresholds (e.g. - X-ray luminosity, $\mathrm{L_{X}}$). This means that 
a steeper WL+SL relation cannot necessarily be attributed to a changing of
cluster shape with radius. The requirement for one to see strong lensing features
around a cluster is a fairly strong selection effect, since chance alignment of
background galaxies are more likely for higher masses with elongation along the
line-of-sight. Nonetheless, we conclude that WL and WL+SL relations remain
inconsistent with those from simulations, if projection is assumed to be the
sole cause of differences in concentration. 

Lastly, in Chapter 4, we explore the relationship between 2-dimensional
measurements of mass and concentration (from clusters collected in Chapter 3)
with the angular structure of the large-scale environment. Galaxy clusters
populate the densest environments in the universe, often times living at the
intersections of large-scale filamentary structures. These super-highways of galaxies
trace out a cosmic web of structure, from which galaxy clusters tend to align
themselves along. If this is in fact true, we would naively expect that having knowledge of
the direction of large-scale structure around clusters may be able to tell us
about the 3-dimensional orientation of the cluster, and therefore break the
degenerate nature of the deprojection of mass along the line-of-sight. We use
1,391,449 spectroscopically confirmed galaxies obtained from the Sloan
Digital Sky Survey (SDSS) Data Release 10 to trace out the large-scale
structure around 92 galaxy clusters within the survey volume. We find
essentially no dependence of cluster mass or concentration measurements with
line-of-sight orientation of the large-scale structure surrounding
them. However, we note that the concentration measurements used for this study
are almost exclusively made with the caustic method and line-of-sight velocity
dispersion, rather than with lensing or X-ray methods. In Chapter 3, we saw evidence
that these particular c-M relations were the most prone to projection
effects. 

Future work will include additional measures of quantifying structure around
clusters, as well as the identification of cluster members. This will allow us
to mitigate the effects of redshift space distortions, and more accurately map
out cluster environments. Ultimately, if a strong connection is found between
the orientation of large scale structure and the over-concentration of clusters, we
would work toward the development of a procedure to constrain 3-dimensional
cluster properties. We also aim to measure the profile properties of a
large, volume limited sample of clusters in each major reconstruction technique
available. This future project would require optical imaging and spectra
(for lensing and galaxy-based methods), in addition to X-ray imaging. With selection
effects essentially marginalized over, this would allow us to calibrate reconstruction
techniques against one another, and truly understand physical biases which may
exist between them.

Throughout this work, we have discussed the details and importance of measuring
density profile properties of galaxy clusters, in both 2- and
3-dimensions. With larger and more sophisticated simulations and major survey
projects like LSST on the horizon, it will soon be possible to study with much
higher precision how shape, orientation, and the large-scale environment alter
our perception of the largest bound structures in the universe.

