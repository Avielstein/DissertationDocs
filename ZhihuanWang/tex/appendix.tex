\chapter{Transtion Rates} \label{ch:mcmc}
This research project focuses on the dimensional dependence of the absorption behavior in semiconductor when interacting with light. Through time-dependent perturbation theory, we find out the transition rate and Fermi’s Golden Rule, then based on the light interaction Hamiltonian, time average Poynting vector and matrix element, derive the absorption coefficient for bulk semiconductor (3-dimension), quantum well (2-dimension) and quantum wire (1-dimension) structure.
Finally, this report will discuss the partial confinement on the electron in the conduction band without the hole confined in the valence band situation.

The Schrodinger equation

The Hamiltonian H can be expressed as:

where  is the unperturbed part Hamiltonian and is time-independent,  is the small perturbation.
The solution to the unperturbed part is assumed known

The time-dependent perturbation is assumed to have the form

Expand the wave function in terms of the unperturbed solution, we find out

gives the probability that the electron is in the state n at time t.
Substituting the expansion for into Schrodinger equation and using (II.3), we have

Taking the inner product with the wave function , and using the orthonormal property,

Therefore, the electron stays at the state i in the absence of any perturbation. The first order solution is 

If we consider the photon energy to be near resonance, either  or , we find the dominant terms:
The absorption coefficient is a strong function of dimensionality. The confining situation of the quantum structure will considerably affect the energy levels, the overlap function of the conduction and valance band envelope function and also the joint optical density of state.
Through the derivation of the absorption coefficient, we can observe the interaction between the light and semiconductor.
The next questions should address:
1. The simulation data for single nanowire abs

%\section{The Model} \label{sec:mcmc_model}


%\section{Conditional Likelihoods} \label{sec:conditionals}
