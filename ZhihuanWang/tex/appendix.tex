\chapter{Time-Dependent Perturbation Theory} \label{ch:rates} 

In this appendix section, we review the time-dependent perturbation theory in
detail as in reference~\cite{Chuang:2009tx}. The method and conclusion will be
used as the fundamental blocks in the derivation of the optical transition
rates in Chapter~\ref{RM}. Starting from the Schr{\"o}dinger equation:

\begin{equation}
  \HAM\Psi(r,t) = - \frac{\hbar}{i}\frac{\partial}{\partial t}\Psi(r,t)
\end{equation}

The Hamiltonian $\HAM$ can be expressed as:
\begin{equation}
  \HAM = \HAM_0 + \HAM^\prime(r,t)
\end{equation}

where $\HAM_0$ is the unperturbed part Hamiltonian and is time-independent,
$\HAM^\prime(r,t)$ is the small perturbation.

The solution to the unperturbed part is assumed known:

\begin{equation}
  \HAM_0\Psi_n(r,t) = - \frac{\hbar}{i}\frac{\partial }{\partial t}\Phi_n(r,t),
\label{eq:unperturbHAM}
\end{equation}

\begin{equation}
  \Phi_n(r,t)=\Phi_n(r)e^{-iE_nt/\hbar}
\end{equation}

The time-dependent perturbation is assumed to have the form:

\begin{equation}
  \HAM = \begin{cases}
    \HAM^\prime(r)e^{-i\omega t} + \HAM^{\prime+}(r)e^{+\omega t}, & t \geq 0 \\
    0, & t < 0
  \end{cases}
\end{equation}

Expand the wave function in terms of the unperturbed solution, we find out
$\Psi(r,t)$:

\begin{equation}
  \Psi(r,t) = \sum_{n}a_n(t)\Phi_n(r)e^{(-iE_{n}t/\hbar)}
\end{equation}

$|a_n(t)|^2$ gives the probability that the electron is in the state n at time
t.

Substituting the expansion for $\Psi$ into Schr{\"o}dinger equation and
using~\ref{eq:unperturbHAM}, we have

\begin{equation}
  \sum_n \frac{da_n(t)}{dt}\Psi_n(r)e^{-iE_{n}t/\hbar} = -\frac{\hbar}{i}\sum_n \HAM^{\prime}(r,t)a_n(t)\Phi_n(r)e^{(-iE_{n}t/\hbar)}
\end{equation}

Taking the inner product with the wave function ${\Phi_m}^\star(r)$, and using
the orthonormal property,

\begin{equation}
  \int{d^3}\bm{r}{{\Phi^\ast}_m}(\bm{r})\Phi_n(\bm{r}) = \delta_{mn}
\end{equation}

We find:

\begin{equation}
\frac{d{a_n(t)}}{dt}=-\frac{i}{\hbar}\sum_{n}a_n(t){\HAM^\prime}_{mn}(t)e^{i\omega_{mn}t}
\end{equation}

where

\begin{eqnarray}
\begin{aligned}
  & {\HAM^\prime}_{mn}(t)=\langle{m}|\HAM^\prime(\bm{r},t)|{n}\rangle \\
  & = \int{\Phi^\ast}_m(\bm{r}){\HAM^\prime}(\bm{r},t)\Psi_n(\bm{r})d^3\bm{r} \\
  & = {\HAM^\prime}_{mn}e^{-i\omega{t}} + {\HAM^\prime}_{mn}e^{+i\omega{t}}
\end{aligned}
\label{eq:HAMintigral}
\end{eqnarray}

\begin{equation}
  {\omega_{mn}}=(E_m-E_n)/\hbar
\end{equation}

and the matrix element is:

\begin{equation}
  {\HAM^\prime}_{mn}(t)=\int{{\Phi^\ast}_m}(\bm{r})\HAM^\prime(\bm{r},t)\Psi_{n}(\bm{r})d^3\bm{r}
\end{equation}

Introducing the perturbation parameter $\lambda$


\begin{equation}
  \HAM=\HAM_0+\lambda\HAM^\prime(\bm{r},t)
\end{equation}

and letting

\begin{equation}
  a_n(t)={a^{(0)}}_n+\lambda{a^{(1)}}_n(t)+\lambda^2{a^{(2)}}_n(t)+\cdots
\end{equation}

we can take the derivative and set $\lambda=1$


\begin{eqnarray}
\begin{aligned}
  & \frac{d{{a^{(0)}}_n(t)}}{dt}=0 \\
  & \frac{d{{a^{(1)}}_n(t)}}{dt}=-\frac{i}{\hbar}\sum_{n}{a^{(0)}_n(t)}{\HAM^\prime}_{mn}(t)e^{i\omega_{mn}t}
\\
  & \frac{d{{a^{(2)}}_n(t)}}{dt}=- \frac{i}{\hbar}\sum_{n}{a^{(1)}_n(t)}{\HAM^\prime}_{mn}(t)e^{i\omega_{mn}t}
\\
\label{eq:probility}
\end{aligned}
\end{eqnarray}

Thus, the zeroth-order solutions for equation~\ref{eq:probility} are constant. Let the electron be at the state i initially

\begin{equation}
  {a^{(0)}}_i(t=0)=1; \quad {a^{(0)}}_m(t)=0, \quad m\neq {i}
\end{equation}

We have the zeroth-order solution

\begin{equation}
  {{a_i}^{(0)}}(t=0)=1; \quad {a^{(0)}}_m(t)=0, \quad m\neq {i}
\end{equation}

Therefore, the electron stays at the state i in the absence of any perturbation. The first order solution is

\begin{eqnarray}
\begin{aligned}
  & \frac{d{a^{(1)}_n}}{dt}=-\frac{i}{\hbar}{\HAM^\prime}_{mn}(t)e^{i\omega_{mn}t} \\
  & = -\frac{i}{\hbar}[{\HAM^\prime}_{mi}e^{-i(\omega_{mi}-\omega)t}+{\HAM^\prime}_{mi}e^{+i(\omega_{mi}+\omega)t}]
\end{aligned}
\end{eqnarray}

If for final state $m=f$; then integrate above equation, we have

\begin{equation}
{a^{(1)}_{f}(t)}=-\frac{i}{\hbar}[{\HAM^\prime}_{fi}\frac{e^{-i(\omega_{mi}-\omega)t}}{\omega_{fi}-\omega}+{\HAM^{\prime+}}_{fi}\frac{e^{+i(\omega_{mi}-\omega)t}}{\omega_{fi}+\omega}]
\end{equation}

If we consider the photon energy to be near resonance, either $\omega\sim\omega_{fi}$ or $\omega\sim-\omega_{fi}$, we find the dominant terms:

\begin{equation}
  {a^{(1)}_{f}(t)}=\frac{4{|{\HAM^\prime}_{fi}|}^2}{\hbar^2}\frac{{sin}^2\frac{t}{2}(\omega_{fi}-\omega)}{{(\omega_{fi}-\omega)}^2} +\frac{4{|{\HAM^\prime}_{fi}|}^2}{\hbar^2}\frac{{sin}^2\frac{t}{2}(\omega_{fi}+\omega)}{{(\omega_{fi}+\omega)}^2}
\end{equation}

where the cross-term has been dropped because it is small compared with either of the above two terms.

When the interaction time is long enough, using approcimation
\begin{equation}
  \frac{{sin}^2(\frac{xt}{2})}{x^2}\rightarrow \frac{\pi{t}}{2}\delta(x)
\end{equation}

Then

\begin{equation}
{|a^{(1)}_f(t)|}^2= \frac{2\pi{t}}{\hbar^2}{|{\HAM^\prime}_{fi}|}^2\delta(\omega_{fi}-\omega)+\frac{2\pi{t}}{\hbar^2}{|{\HAM^\prime}_{fi}|}^2\delta(\omega_{fi}+\omega)
\end{equation}

The transition rate should be, after using the property of

\begin{equation}
  \delta(\hbar\omega)=\frac{\delta(w)}{\hbar}
\end{equation}

\begin{eqnarray}
  & W_{if} = \frac{d{|a^{(1)}_f(t)|}^2}{dt} \\ 
  & = \frac{2\pi}{\hbar}{|{\HAM^\prime}_{fi}|}^2\delta(E_{f}-E_{i}-\hbar\omega)+\frac{2\pi}{\hbar}{|{\HAM^\prime}_{fi}|}^2\delta(E_{f}-E_{i}+\hbar\omega)
\end{eqnarray}

where $E_f = E_i+ \hbar\omega$ represents the absorption of a photon by an electron, and $E_f = E_i - \hbar\omega$ corresponds with the emission of a photon.

\chapter{Partial Confinement on the Electron in Conduction
Band}\label{partialconfinement}

If the one-dimensional confinement only apply to the electrons in the
conduction band, i.e., the holes in the valance band are free to move as
in the bulk semiconductor, the wavefunction in the conduction band and
valance band will change accordingly. The overlap of the conduction and
valance band envelope function will no longer exist.

Within a two-band model, the Bloch wave functions can be described by

\begin{equation}
\Psi_{a}\left( \bm{r} \right) = u_{v}(\bm{r})\frac{e^{i\bm{k}_{\bm{t}} \cdot \rho}}{\sqrt{L_{z}}}
\end{equation}

for a hole wave function in the heavy-hole or a light-hole subband m.
and

\begin{equation}
\Psi_{b}\left( \bm{r} \right) = u_{c}(\bm{r})\frac{e^{i\bm{k}_{\bm{t}} \cdot \rho}}{\sqrt{L_{z}}}\Phi_{n}(x,y)
\end{equation}

for an electron in the conduction subband n. The momentum matrix element
\(\bm{p}_{\text{ba}}\) is given by

\begin{equation}
\bm{p}_{\text{ba}}\bm{=}\left\langle \Psi_{b} \middle| \bm{p} \middle| \Psi_{a} \right\rangle \approx \left\langle u_{c} \middle| \bm{p} \middle| u_{v} \right\rangle\delta_{k_{t},k_{t}^{'}}\ I_{\text{en}}
\end{equation}

where

\begin{eqnarray}
  & I_{\text{en}} = \int_{- \infty}^{+ \infty}{\text{dxdy}\Phi_{n}^{*}\left( x,y \right)} \nonumber \\
  & = \int_{- \infty}^{+ \infty}{dxdy \cdot const \times e^{- \alpha^{2}y^{2}}\mathcal{H}_{n_{1}}(\alpha y)sin\frac{\text{πx}n_{2}}{L_{x}}}
\end{eqnarray}

Here introduce the notations

\begin{equation}
\alpha = \frac{m_{e}^{*}\omega}{\hbar}\ , \quad
\mathcal{H}_{n}\left( y \right) = {( - 1)}^{n}e^{y^{2}}\frac{d^{n}}{dy^{n}}e^{{- y}^{2}}
\end{equation}

where \(\mathcal{H}_{n}\) are the Hermite
polynomials~\cite{Mitin:1999vs},
\(n_{1}\text{\ and\ }n_{2}\) are two quantum numbers.

There is no overlap of the conduction and valence band envelope
functions and the \textbf{K}-Selection rule also applied. The energy
levels, which arise in quantum wires, are strongly dependent on the form
of the confining potentials. And the additional confinement of electrons
leads to an increase of the lowest energy level.

Take into account the quantization of the electron and hole energies
\(E_{a}\text{\ and\ }E_{b}\)

\begin{equation}
E_{a} = E_{\text{hm}} - \frac{\hbar^{2}\bm{k}_{\bm{t}}^{2}}{2m_{h}^{*}}
\end{equation}

\begin{equation}
E_{b} = {E_{g} + E}_{\text{en}} + \frac{\hbar^{2}\bm{k}_{\bm{t}}^{2}}{2m_{e}^{*}}
\end{equation}

And \(E_{\text{hm}} < 0\),

\begin{equation}
E_{b} - E_{a} = E_{\text{hm}}^{\text{en}} + E_{t},
E_{t} = \frac{\hbar^{2}\bm{k}_{\bm{t}}^{2}}{2m_{e}^{*}}
\end{equation}

where

\begin{equation}
E_{\text{hm}}^{\text{en}} = {E_{g} + E}_{\text{en}} - E_{\text{hm}}
\end{equation}

is the band edge transition energy (\(\bm{k}_{\bm{t}} = 0\)).
The summations over the quantum numbers
\(\bm{k}_{a}\text{\ and}\bm{\ }\bm{k}_{b}\)become summations
over (\(\bm{k}_{\bm{t}}^{\bm{'}},m\)) and
(\(\bm{k}_{\bm{t}},n\)). Noting in the matrix element
\(\bm{k}_{\bm{t}}\bm{=}\bm{k}_{\bm{t}}^{\bm{'}}\)

\begin{equation}
\alpha\left( \hbar\omega \right) = C_{0}\sum_{n}^{}\left| I_{\text{en}} \right|^{2}\frac{2}{V}\sum_{\bm{k}_{t}}^{}\left| \hat{e}\bm{\cdot}\bm{p}_{\text{cv}} \right|^{2}\delta(E_{\text{hm}}^{\text{en}} + E_{t} - \hbar\omega)(f_{v}^{m} - f_{c}^{n})
\end{equation}

Similarly, for this quasi-one dimensional case, assume the
one-dimensional joint density of states also apply

\begin{equation}
\frac{2}{V}\sum_{\bm{k}_{t}}^{}{= \frac{2L_{z}}{V}}\int_{}^{}\frac{d\bm{k}_{\bm{t}}}{2\pi} = \frac{1}{\pi L_{x}L_{y}}\int_{0}^{\infty}\frac{\bm{1}}{\bm{k}_{\bm{t}}}d\bm{k}_{\bm{t}}\bm{=}\int_{0}^{\infty}{dE_{t}}\rho_{r}^{1D}
\end{equation}

\begin{equation}
\rho_{r}^{1D} = \frac{\left( m_{r}^{*} \right)^{\frac{3}{2}}}{\text{πℏ}m_{e}^{*}L_{x}L_{y}}\frac{\bm{1}}{\sqrt{\bm{(}\hbar\omega - E_{g}\bm{)}}}
\end{equation}

where \(L_{z}L_{x}L_{y} = V\), \({L_{x},L_{y,}L}_{z}\) are effective
period of the quantum wire along different directions, \(L_{z}\) along
the axial of the quantum wire, and V is a volume of a period. The delta
function gives the contribution at
\(E_{\text{hm}}^{\text{en}} + E_{t} = \hbar\omega\), and the
absorption edges occur at \(\hbar\omega = E_{\text{hm}}^{\text{en}}\).
For an unpumped semiconductor, \(f_{v}^{m} = 1\ and\ f_{c}^{n} = 0\), we
have the absorption spectrum at thermal equilibrium
\(\alpha_{0}\left( \hbar\omega \right)\)

\begin{equation}
\alpha_{0}\left( \hbar\omega \right) = C_{0}\sum_{n}^{}\left| I_{\text{en}} \right|^{2}\left| \hat{e}\bm{\cdot}\bm{p}_{\text{cv}} \right|^{2}\rho_{r}^{1D}H(\hbar\omega - E_{\text{hm}}^{\text{en}})
\end{equation}

Because the integration of the delta function gives the step function,
shown as $H$ or the Heaviside step function,
\(H\left( x \right) = 1\ for\ x > 0,\ and\ 0\ for\ x < 0\). The
summation of \(I_{\text{en}}\)becomes the integral of conduction band
electron envelope function, using an infinite wire model, and the
absorption spectrum is

\begin{equation}
  \alpha_{0}\left( \hbar\omega \right) = C_{0}{\sum_{n}^{}\left| I_{\text{en}} \right|^{2}\left| \hat{e}\bm{\cdot}\bm{p}_{\text{cv}} \right|}^{2}\frac{\left( m_{r}^{*} \right)^{\frac{3}{2}}}{\pi\hbar{m_{e}}^{*}L_{x}L_{y}}\frac{\bm{1}}{\sqrt{\bm{(}\hbar\omega - E_{g}\bm{)}}}
\end{equation}

\begin{equation}
C_{0} = \frac{\pi e^{2}}{n_{r}\varepsilon_{0}cm_{0}^{2}\omega}
\end{equation}

We can see that the factors containing \(A_{0}^{2}\) are canceled
because the linear optical absorption coefficient is independent of the
optical intensity.

\chapter[Lasing Modeling]{Semiconductor Laser Modeling}
\label{sec:model}

In this section, we are trying to delve into the mechanics of how an injected
current actually results in an optical output in a semiconductor heterojunction
laser by providing a systematic derivation of the dc light-current
characteristics. First, the rate equation for photon generation and loss in a
laser cavity is developed. This shows that only a small portion of the
spontaneously generated light contributes to the lasing mode. Most of it comes
from the stimulated recombination of carriers. All of the carriers that are
stimulated to recombine by light in a certain mode contribute more photons to
that same mode. Thus, the stimulated carrier recombination/photon generation
process is a gain process. We find the threshold gain for lasing which is the
gain necessary to compensate for cavity losses. The threshold current is the
current required to reach this gain.

For electrons and holes in the active region of a diode laser, only a fraction,
$\eta_i$, of injected current will contribute to the generation of carriers. We
assumed the active regions that are undoped or lightly doped, so that under
high injection levels, charge neutrality applies and the electron density
equals the hole density (i.e., $N = P$ in the active region). Thus, we can
greatly simplify our analysis by specifically tracking only the electron
density, N.

We start with the governing equations of carrier density and photon density in
the active region which is governed by a dynamic process.

\begin{eqnarray}
\begin{aligned}
  & \frac{dN}{dt} = \frac{\eta_{i}I}{qV} - \frac{N}{\tau} - R_{st},
  \\
  & \frac{dN_p}{dt} = {\Gamma}v_g{g}N_p + \Gamma\beta_{sp}R_{sp} - \frac{N_p}{\tau_p},
\end{aligned}
\label{eq:NgoverningEq}
\end{eqnarray}

where $\beta_{sp}$ is the spontaneous emission factor, defined as the
percentage of the total spontaneous emission coupled into the lasing mode. And
it is just the reciprocal of the number of available optical modes in the
bandwidth of the spontaneous emission for uniform coupling to all modes. The g
is the incremental gain per unit length.

The first term of Eq.~\ref{eq:NgoverningEq} is the rate of injected electrons
$G_{gen} = {\Gamma_{i}I}/{qV}$, ${\Gamma_{i}I}/{q}$ is the number of electrons
per second being injected into the active region, where V is the volume of the
active region. The rest terms are the rate of recombining of electrons per unit
volume in the active region. There are several mechanisms should be considered,
including a spontaneous recombination rate, $R_{sp}$, a nonradiative
recombination rate, $R_{nr}$, a carrier leakage rate, $R_l$ and a net
stimulated recombination, $R_{st}$, including both stimulated absorption and
emission. Which looks like:

\begin{equation}
  R_{rec} = R_{sp} + R_{nr} + R_{l} + R_{st}
\end{equation}

The first three terms on the right refer to the natural or unstimulated carrier
decay processes. The fourth one, $R_{st}$, require the presence of photon.

The natural decay process can be described by a carrier lifetime, $\tau$. In
the absence of photon or a generation term, the rate equation for carrier decay
is $dN/dt = -N/\tau$, where $N/\tau = R_{sp} + R_{nr} + R_{l}$.

The natural decay rate can also be expressed in a power series of the carrier
density, N. We can also rewrite $R_{rec} = BN^2 + (AN + CN^3) + R_{st}$. Where
$R_{sp} \sim BN^2$ and $R_{nr} + R_{l} \sim (AN + cN^3)$. The coefficient $B$
is the bimolecular recombination coefficient, and it has a magnitude, $B \sim
10^{-10} cm^3/s$ for most AlGaAs and InGaAsP alloys of interest.

When a laser is below threshold, in which the gain is insufficient to
compensate for cavity losses, the generated photons do not receive net
amplification. The spontaneous photon generation rate per unit volume is
exactly equal to the spontaneous electron recombination rate, $R_{sp}$, because
an electron-hole pair will emit a photon when they recombine radiatively.
Under steady-state conditions ($dN/dt = 0$), the generation rate equals the
recombination rate with $R_{st} = 0$.

\begin{equation}
  \frac{\eta_{i}I}{qV} = R_{sp} + R_{nr} + R_{l}
\end{equation}

The spontaneously generated optical power, $P_{sp}$, is obtained by multiplying
the number of photons generated per unit time per unit volume, $R_{sp}$, by the
energy per photon, $hv$, and the volume of the active region, V.

\begin{equation}
  P_{sp} = h{\upsilon}VR_{sp} = \eta_i\eta_r\frac{h\upsilon}{q}I
\end{equation}

The main photon generation term above threshold is $R_{st}$. Electron-hole pair
is stimulated to recombine, another photon is generated. But since the cavity
volume occupied by photons, $V_p$, is usually larger than the active region
volume occupied by electrons, V, the photon density generation rate will be
$[V/V_p]R_{st}$, not just $R_{st}$. The electron-photon overlap factor,
$[V/V_p]$, is generally referred to as the confinement factor, $\Gamma$.

\section[Steady-State Gain]{Threshold or Steady-State Gain in Lasers} \label{sec:steadystate}

The cavity loss can be characterized by a photon decay constant or lifetime,
$\tau_p$, and the gain necessary to overcome losses, and thus reach threshold.
By assuming steady-state conditions (\ie $dN_p/dt = 0$), and solving for this
steady-state or threshold gain, $g_{th}$, where the generation term equals the
recombination term for photons. We assume only a small fraction of the
spontaneous emission is coupled into the mode (\ie $\beta_{sp}$ is quite
small), then the second term can be neglected, and we have the solution:

\begin{equation}
  \Gamma{g_{th}} = \frac{1}{v_g\tau_p} = <\alpha_i> + \alpha_m
\end{equation}

The product, $\Gamma{g_{th}}$, is referred to as the threshold modal gain
because it now represents the net gain required for the mode as a whole, and it
is the mode as a whole that experiences the cavity loss. $<\alpha_i>$ is the
average internal loss, and $\alpha_m$ is the mirror loss if we considered an
in-plane wave laser.

The optical energy of a nano-cavity laser propagates in a dielectric waveguide
mode, which is confined both transversely and laterally as defined by a
normalized transverse electric field profile, $U(x,y)$. In the axial direction
this mode propagates as $exp^{(-j\beta z)}$, where $\beta$ is the complex
propagation constant, which includes any loss or gain. Thus, the time- and
space-varying electric field can be written as

\begin{equation}
  \xi = \hat{e}_{y}E_{0}U(x,y)e^{j(\omega t- \beta z)}
\end{equation}

where $\hat{e}_y$ is the unit vector indicating TE polarization and $E_0$ is
the magnitude of the field. The complex propagation constant, $\beta$, includes
the incremental transverse modal gain, $<g>_{xy}$ and internal modal loss,
$<\alpha_i>_{xy}$. If we consider a Fabry-Perot laser with the propagating mode
is reflected by end mirrors, and the reflection coefficients are $r_1$ and
$r_2$. respectively. In addition, the mean mirror intensity reflection
coefficient, $R = r_1r_2$.

Define the mirror lass as $\alpha_m$

\begin{equation}
  \alpha_m \equiv \frac{1}{L}\ln(\frac{1}{R})
\end{equation}

The photon decay lifetime is given by,

\begin{equation}
  \frac{1}{\tau_p} = \frac{1}{\tau_i} + \frac{1}{\tau_m} = v_g(<\alpha_i> + \alpha_m)
\end{equation}

Thus, we can also write

\begin{equation}
  \Gamma g_{th} = <\alpha_i> + \alpha_m = \frac{1}{v_g\tau_p}
\end{equation}

\section[Threshold Output Power]{Threshold Current and Output Power}
\label{sec:threshold_current_and_power_out_versus_current}

We construct together a below-threshold and an above-threshold characteristic
to illustrate the output power versus input current for a normal diode laser.
The first step is to use the below-threshold steady-state carrier rate
equation,

\begin{equation}
  \frac{\eta_{i}I_{th}}{qV} = {(R_{sp} + R_{nr} + R_{l})}_{th} = \frac{N_{th}}{\tau}
  \label{eq:thresholdsteadyrate}
\end{equation}

Then, recognizing that $(R_{sp} + R_{nr} + R_{l}) = AN + BN^2 +CN^3$ depends
monotonically on $N$, we observe from $N(I > I_{th}) = N_{th}$ that above
threshold $(R_{sp} + R_{nr} + R_{l})$ will also clamp at its threshold value,
given by Eq.~\ref{eq:thresholdsteadyrate}. Thus, we can substitute
Eq.~\ref{eq:thresholdsteadyrate} into the carrier rate equation,
Eq.~\ref{eq:NgoverningEq} to obtain a new above threshold carrier rate
equations,

\begin{equation}
  \frac{dN}{dt} = \eta_i \frac{(I - I_{th})}{qV} - v_{g}gN_p,~~~   (I > I_{th})
\end{equation}

We also calculate a steady-state photon density above threshold where $g =
g_{th}$,

\begin{equation}
  N_p = \frac{\eta_i (I - I_{th})}{qv_{g}g{th}V}~~~   (steady~ state)
\end{equation}

To obtain the power out, we first construct the stored optical energy in the
cavity, $E_{os}$, by multiplying the photon density, $N_p$, by the energy per
photon, $hv$, and the cavity volume, $V_p$. That is $E_{os} = N_phvV_p$. Then,
we multiply this by the erngy loss rate through the mirrors, $v_g\alpha_m =
\frac{1}{\tau_m}$, to get the optical power output from the mirrors,

\begin{equation}
  P_0 = v_g\alpha_{m}N_phvV_p
\end{equation}

Substituting from , and using $\Gamma = V/V_p$,

\begin{equation}
  P_0 = \eta_i(\frac{\alpha_m}{<\alpha_i> + \alpha_m})\frac{hv}{q}(I - I{th}),~~~(I > I_{th})
\end{equation}

The output power below-threshold $(I < I_{th})$ can be approximated by
neglecting the stimulated emission term and solving for $N_p$ under
steady-state conditions.

\begin{equation}
N_p = \Gamma\beta_{sp}R_{sp}\tau{p}~~~(I < I_{th})
\end{equation}

and

\begin{equation}
  P_0(I < I_{th}) = \eta_r\eta_i\left(\frac{\alpha_m}{<\alpha_i> + \alpha_m}\right)\frac{hv}{q}\beta_{sp}I,
\end{equation}

We can get the threshold carrier density:

\begin{equation}
  N_{th} = N_{tr}e^{g_{th}/g_{0}N} = N_{tr}e^{(<\alpha_i> + \alpha_m)/\Gamma{g_{0}}N}
\end{equation}

Using the polynomial fit for the recombination rates, and recognizing that for
the best laser material the recombination at threshold is dominated by
spontaneous recombination, we have, $I_{th}\cong B{N_{th}}^2qV/\eta_i$, Thus

\begin{equation}
  I_{th} {\cong} \frac{qVB{N_{tr}}^2}{\eta_i}e^{(<\alpha_i> + \alpha_m)/\Gamma{g_0}N}
\end{equation}

where for most $\uppercase\expandafter{\romannumeral3} -
\uppercase\expandafter{\romannumeral5}$ materials the bimolecular recombination
coefficient, $B \sim 10^{-10} cm^3/s$.

For a multiple quantum-well (MQW) lasers, we have to multiply the single-well
confinement factor, $\Gamma_1$, and volume, $V_1$, by the number of wells,
$N_w$.

\begin{equation}
  I_{thMQW} {\cong} \frac{qN_{w}V_{1}B{N_{tr}}^2}{\eta_i}e^{2(<\alpha_i> + \alpha_m)/{N_w\Gamma_{1}g_{0}N}}
\end{equation}

\chapter{MEEP Simulation Code} \label{ch:meepcode} 
\section{Cylindrical Core-Shell Nanowire}

%\SyntaxHighLighting{}
\lstinputlisting[language=Scheme]{snippet/sourcecode/CylindCS.ctl}

\section{Hexagonal Core-Shell Nanowire}
%\SyntaxHighLighting{}
\lstinputlisting[language=Scheme]{snippet/sourcecode/HexCS3D.ctl}

\chapter{Gain Spectrum and Threshold Calculation Matlab Code} \label{ch:matlabcode} 
%\SyntaxHighLighting{}
\lstinputlisting[language=matlab]{snippet/sourcecode/gain1band.m}

\chapter[NW Lasing Modeling]{Semiconductor nanowire laser modeling Matlab Code} \label{ch:NWLaserMcode} 
\section{FDTD Simulation Results Processing}
%\SyntaxHighLighting{}
\lstinputlisting[language=matlab]{snippet/sourcecode/Mode_CSNW_FDTD.m}
\section{Steady State Rate Calculation}
%\SyntaxHighLighting{}
\lstinputlisting[language=matlab]{snippet/sourcecode/CSNW_Laser.m}
