\chapter{Introduction}

The development of sophisticated growth techniques for layered semiconductor
structures has stimulated a large body of new work in semiconductor physics
over the last fifteen years or so. Undoubtedly, much of this interest was
further stimulated by the possibility of novel physics and applications in
electronic transport. New physical discovered in inversion channels and
heterostructures, and the first heterostructure electronic devices, such as
modulation-doped field-effect transistors and heterojuntion bipolar
transistors, are now being commercially exploited. Linear optical spectroscopic
techniques, such as absorption, luminescnence and modulation spectroscopy, have
for a long time been important tools in understanding the basic physics of
semiconductor materials. Also over the last fifteen years or so, semiconductor
optical and optioelectronic properties have become of increasing technological
importance in their own right. The ever-growing application of semiconductor
diode lasers and related optoelectronic technology in communications and
consumer products has helped to give yet further impetus to research on
semiconductor optical properties. 

The successes of semiconductor optoelectronics and promising physical
mechanisms and novel devices using quantum-confined structures have,
furthermore, enlivened the debate over possible applications of optics for
other functions such as logic and switching in communications and computation.

It is important to emphasize at the outset that quantum confinement and produce
not only quantitative but also qualitative differences in physics from that in
bulk structures, which is of course another major motivation for the interest
in them. There are many examples of these differences. The optical absorption
spectrum breaks up into a series of steps associated with the quantum-confined
electron and hole levels. Excitonic effects become much stronger because of the
quantum confinement, giving clear absorption resonances even at room
temperature. The relative importance of direct Coulomb screening and exchange
effects is quite different in quantum wells (the Coulomb screening is
relatively much weaker), giving very different optical saturation behaviour.

\section{Background} \label{sec:AD}


%\subsection{$\alpha$--Disk Model}


%\subsection{Eddington Luminosity} \label{Eddington_fraction}


%\section{Spectral Energy Distributions}


\section{Problem Statement} \label{sec:dust_intro}

\section{Literature Review} \label{sec:intro_MBH}


\section{Outline of Chapters}

This thesis is structured as follows. The data used for this analysis are presented in Chapter~\ref{data}.  After introducing the corrections and data analysis methods, Chapter~\ref{SEDs} presents our findings for individual and mean bolometric corrections.  Chapter~\ref{Dust} presents our methods and findings for estimating dust reddening associated with a subset of our quasars.  The statistical methods used in this chapter are outlined in more detail in Appendix~\ref{ch:mcmc}. In Chapter~\ref{BH}, we apply the corrections from Chapter~\ref{Dust} to our SEDs and use estimated black hole masses to study changes in the SEDs with the properties of the central black hole. Finally, we present our conclusions in Chapter~\ref{conclusions}. Throughout this work we use a $\Lambda$CDM cosmology with $H_0=71$ km s$^{-1}$ Mpc$^{-1}$, $\Omega_\Lambda = 0.734$, and $\Omega_m = 0.266$, consistent with the {\em Wilkinson Microwave Anisotropy Probe} 7 cosmology \citep{Jarosik:2011}.

