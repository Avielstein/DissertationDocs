\chapter{Introduction}


\section{Background} \label{sec:AD}


%\subsection{$\alpha$--Disk Model}


%\subsection{Eddington Luminosity} \label{Eddington_fraction}


%\section{Spectral Energy Distributions}


\section{Problem Statement} \label{sec:dust_intro}

\section{Literature Review} \label{sec:intro_MBH}


\section{Outline of Chapters}

This thesis is structured as follows. The data used for this analysis are presented in Chapter~\ref{data}.  After introducing the corrections and data analysis methods, Chapter~\ref{SEDs} presents our findings for individual and mean bolometric corrections.  Chapter~\ref{Dust} presents our methods and findings for estimating dust reddening associated with a subset of our quasars.  The statistical methods used in this chapter are outlined in more detail in Appendix~\ref{ch:mcmc}. In Chapter~\ref{BH}, we apply the corrections from Chapter~\ref{Dust} to our SEDs and use estimated black hole masses to study changes in the SEDs with the properties of the central black hole. Finally, we present our conclusions in Chapter~\ref{conclusions}. Throughout this work we use a $\Lambda$CDM cosmology with $H_0=71$ km s$^{-1}$ Mpc$^{-1}$, $\Omega_\Lambda = 0.734$, and $\Omega_m = 0.266$, consistent with the {\em Wilkinson Microwave Anisotropy Probe} 7 cosmology \citep{Jarosik:2011}.

