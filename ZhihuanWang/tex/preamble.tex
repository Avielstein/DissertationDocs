\begin{preamble}

\iffinal{}{\newpage}

\begin{DUTdedications}
%
\vspace*{\fill}
%
\begin{center}
\setstretch{2}
\begin{minipage}{8 cm}
\begin{center}
\hrulefill\\
This thesis is dedicated to my family. Their unconditional support and love was the foundation of success for my graduate study. I want you to know that I love you so much and this thesis was only possible thanks to you. 
\hrulefill
\vspace{6em}
\end{center}
\end{minipage}
\end{center}
\vspace*{\fill}
\end{DUTdedications}

\iffinal{}{\newpage}

\begin{acknowledgments}
%Over the past six years I have received support and encouragement form many
    %people.  
This work would not have been possible without the support of my advisor, Dr.
Bahram Nabet. His guidance helped to shape and provided much needed focus to my
work.  I would like to thank my dissertation committee of Dr. Timothy Kurzweg,
Dr. Baris Taskin, Dr. Nagarajan Kandasamy, and Dr. Goran Karapetrov for their
support and insight throughout my research.

I would also like to thank my friends in the Photonics Lab for all the help and
support they provided, and especially for providing such a friendly and
awesome place to work and study at. Weston Aenchbacher and Shrenik Vora have spent countless hours listening to me talk about
my research, helping me flesh out my ideas. And a special thanks to Jiajia Liu
for proofreading all of my work and for encouraging me throughout my graduate
work.  

\end{acknowledgments}

\iffinal{}{\newpage}

\tableofcontents 
\iffinal{}{\newpage}

\listoftables
\iffinal{}{\newpage}

\listoffigures 
\iffinal{}{\newpage}

\begin{abstract}
%\ifdaring{\setstretch{1.3}}{}
\ifdaring{\setstretch{1.3}}{\setstretch{1.6}}

Semiconductor nanowires have been used in a variety of passive and active
optoelectronic devices including waveguides, photodetectors, solar cells, LEDs,
Lasers, sensors, and optical antennas.  We  review  the  optical  properties
of  these  nanowires in  terms  of  absorption, guiding,  and  radiation  of
light,  which  may  be  termed  light  management.  Analysis  of  the
interaction   of   light   with   long   cylindrical   structures   with
sub-wavelength   diameters identifies  radial  resonant  modes,  such  as
Leaky  Mode  Resonances,  or  Whispering  Gallery modes.  The  two-dimensional
treatment  should  incorporate  axial  variations inDzvolumetric modesdz which
have  so  far  been  presented  in  terms  of  Fabry-Perot, and  Helical
resonance modes. We report on FDTD simulations with the aim of identifying the
dependence of these modes  on:    geometry  (length,  width),  tapering,  shape
(cylindrical,  hexagonal),  core-shell versus  core-only,  and  dielectric
cores  with  semiconductor  shells. This  demonstrates  how NWs form excellent
optical cavities without the need for top and bottom mirrors. However,
optically  equivalent  structures  such  as  hexagonal  and  cylindrical  wires
can  have  very different   optoelectronic   properties   meaning   that
light   management   alone   does   not sufficiently  describe  the  observed
enhancement  in  upward  (absorption)  and  downward transitions  (emission)
of  light  in  nanowires;  rather,  the  electronic  transition  rates  should
be considered.  We discuss this Dzrate-managementdzscheme showing its strong
dimensional dependence,  making  a  case  for  photonic  integrated  circuits
that  can  take  advantage  of  the confluence of the desirable optical and
electronic properties of these nanostructures.  
\end{abstract}

\iffinal{}{\newpage}

\end{preamble}
