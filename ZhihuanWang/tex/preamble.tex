\begin{preamble}

\iffinal{}{\newpage}

\begin{DUTdedications}
%
\vspace*{\fill}
%
\begin{center}
\setstretch{2}
\begin{minipage}{8 cm}
\begin{center}
\hrulefill\\

This thesis is dedicated to my family. Their unconditional support and love was
the foundation of success for my graduate studies. I want you to know that I love
you so much and this thesis was only possible thanks to you.

\hrulefill
\vspace{6em}
\end{center}
\end{minipage}
\end{center}
\vspace*{\fill}
\end{DUTdedications}

\iffinal{}{\newpage}

\begin{acknowledgments}

This dissertation summarized the research work I have accomplished during my
graduate study in Drexel University. Over all these years, I obtained
tremendous help from all the people around and I cannot complete the dissertation
without expressing my gratitude to them.

This work would not have been possible without the support of my advisor, Dr.
Bahram Nabet. His guidance helped to shape and provided much needed focus to my
work. I would also like to thank my dissertation committee of Dr. Timothy
Kurzweg, Dr. Baris Taskin, Dr. Nagarajan Kandasamy, and Dr. Goran Karapetrov
for their support and suggestions.

Many thanks to our generous collaborators, Dr. Adriano Cola, Dr. Anna Persano,
Dr. Paola Prete at IMM-NCR in Italy, Dr. Nico Lovergine at University of
Salento in Italy and Dr. Marc Currie from Navel Research Laboratory in
Washington D.C. for their tremendous efforts in fabrication of the AlGaAs/GaAs
core-shell nanowires and electro-optically sampling, micro-photoluminescence
measurements of the devices. Although, I have not had the opportunity of
meeting them yet, it is still a great pleasure to communicate with them in
Email and I appreciate their valuable thoughts and discussions in many of the
experimental results. I would also like to dedicate many thanks to Dr. Fernando
Camino, Dr. Aaron Stein, Dr. Chang-Yong Nam, Dr. Mircea Cotlet and many other
staffs at CFN of Brookhaven National Laboratory in Long Island for their
assistance and suggestions in characterizing the devices.

I would also like to thank my friends in the Photonics Lab for all the help and
support they provided, and especially for providing such a friendly and awesome
place to work and study at. Pouya Dianat, Weston Aenchbacher and Shrenik Vora
have spent countless hours listening to me talk about my research, helping me
flesh out my ideas. And a special thanks to Jiajia Liu for accompanying me
during all these years and bringing countless happiness to my life.

Finally, my parents deserve my most sincere gratitude. I appreciate that you
spent one or two months each year to be in the States with me, and that you
always believed in, encouraged and loved me even when you lived on the other
side of the earth.

\end{acknowledgments}

\iffinal{}{\newpage}

\tableofcontents 
\iffinal{}{\newpage}

\listoftables
\iffinal{}{\newpage}

\listoffigures 
\iffinal{}{\newpage}

\begin{abstract}
%\ifdaring{\setstretch{1.3}}{}
\ifdaring{\setstretch{1.3}}{\setstretch{1.6}}

Semiconductor nanowires have been used in a variety of passive and active
optoelectronic devices including waveguides, photodetectors, solar cells, LEDs,
Lasers, sensors, and optical antennas. We examine GaAs/AlGaAs core-shell
nanowires (CSNWs) grown on both GaAs and Si substrates by vapor-liquid-solid
(VLS) method followed by MOCVD. These nanowires show extremely enhanced optical
properties in terms of absorption, guiding, radiation of light, and even
lasing. Analysis of the interaction of light with cylindrical and hexagonal
structures with sub-wavelength diameters identifies both transverse and
longitudinal plane modes which we generalize to volumetric resonant modes,
importantly without the need for vertical structures such as Bragg mirrors
commonly used in vertical cavity surface emitting lasers (VCSEL's). We report
on FDTD simulations with the aim of identifying the dependence of these modes
on geometry (length, width), tapering, shape (cylindrical, hexagonal),
core-shell versus core-only, and dielectric cores with semiconductor shells.
This demonstrates how NWs form excellent optical cavities without the need for
top and bottom mirrors. However, optically equivalent structures such as
hexagonal and cylindrical wires can have very different optoelectronic
properties meaning that light management alone does not sufficiently describe
the observed enhancement in upward (absorption) and downward transitions
(emission) of light in nanowires, rather, the electronic transition rates
should be considered. Using Fermi's Golden Rule in interaction of light and
matter, we discuss how the transition rates change due to electronic wave
function and identify three factors, namely, oscillator strength, overlap
functions, and the joint optical density of states(JODS), explicitly
contributing to the transition rates with strong dependence on dimensionality.
We apply these results to the study of lasing in as-grown CSNW on Si \& GaAs
and discuss how these subwavelength structures can have enhanced optical gain
and quantum efficiency compared to their bulk counterparts despite their large
$> 200nm$ geometries. These results and findings will further facilitate the
design and optimization of sub-micron scale optoelectronic devices. In
conclusion, we make a case for photonic integrated circuits that can take
advantage of the confluence of the desirable optical and electronic properties
of these nanostructures.

\end{abstract}

\iffinal{}{\newpage}

\end{preamble}
