\chapter{Light confinement in sub-wavelength nano-structure} \label{LM}


\section{Light and Nanowire} \label{corrections}


\subsection{Leaky Mode Resonance}
\label{sec:hydrogen}
Interaction of light with a dielectric or metallic cylindrical medium is analyzed by solving Maxwell's equations with the appropriate boundary conditions in the classical waveguide theory [80] which leads to highly confined modes in optical fibers and microscale dielectric resonators. In an infinitely long cylinder, even at deep sub-wavelength diameters this results in a “characteristic equation” the solution to which are the transverse magnetic (TM) and transverse electric (TE) resonant modes. We can define the electromagnetic modes of localized resonators as time-harmonic solutions of the form  to the source-free Maxwell equations. This solution shows that the longitudinal field component distributes outside the NW, and is in resonance with the natural modes, such as TE11, TM02, etc., supported by the NW. These modes have been termed leaky-mode resonances (LMR) [81, 82], and provide an intuitive tool to facilitate the understanding and optimization of the resonance effect in such nano-structures. 
We replicate these results using MEEP, a widely used open-source finite-difference time-domain (FDTD) simulation package [83], to identify how light is confined in an infinitely long GaAs nanowire. The top row of Fig. 1 shows several configurations of TM LMR modes for a NW with diameter of 220nm, with excitation single wavelength light being incident parallel to the NW axis. The blue and red color codes represent the polarization of the electric fields. The TE modes are primarily identical to the TM modes shown here with the electric and magnetic fields exchanged. If the light is incident with an arbitrary angle, then so-called hybrid HE and EH leaky modes will be excited instead of the pure TE or TM mode.  The bottom row of Fig. 1 shows the directional energy flux density of the electromagnetic field, the Poynting vector, at different time frames with light being incident perpendicular to the NW axis from the right side. The light is seen to propagate from the right and then mostly remain confined at the left part of the cylinder. It is notable that in either case the light energy is spatially distributed along the cross section of the wire but, as expected from a 2D treatment does not vary axially. Figure 1 demonstrates that the LMR can gently confine light within subwavelength semiconductor nano-structures, similar to the intuitive ray-optics picture of multiple total internal reflections from the periphery of the cylinder. As shown by in Ref. [81], these LMRs depend on the radius and the height of the dielectric, which allows ‘light engineering’ of the nanowires so as to increase its absorption efficiency at pre-determined wavelength, e.g., to maximize absorption of sunlight spectrum for  higher efficiency solar cells, or to radiate as optical antennas. 

\subsection{Whispering Gallery Modes}
\label{sec:host}
Infinitely long cylindrical or hexagonal NW structures can also support Whispering Gallery (WG) modes [78, 79, 84-90]. To calculate the resonant WGMs, Maxwell’s equations have to be solved numerically [91] taking into consideration the spectral dependence of the material of interest’s index of refraction. However, we can deduce a simple plane-wave model from theoretical derivations, and the relationship between resonance wavelength λ and the corresponding mode serial number N can be obtained [92]. The WG modes can also reflect and confine light in the (subwavelength) nanostructure by total internal reflection from the curvature of the structure boundaries. However, a light wave can interfere with itself only when having completed one full circulation within the resonator, which means only the light with one or multiple wavelengths are allowed to perform multiple circulations generating a standing wave. Figure 2 from reference 85 shows near-field intensity patterns of low-order TM polarized hexagonal WGMs for n=1 and refractive index =2.1. Each mode pattern is labeled by its respective mode number m (lower right number) and its symmetry class (upper right symbol). 
For comparison, four mode patterns of the circular cavity are given in the upper left and lower right together with their angular mode number. We again observe the radial spatial dependence of light intensity. Furthermore, the low order WG modes of hexagonal NWs are essentially similar to the cylindrical ones, but for higher order modes additional features will arise on the facets of the hexagonal NWs [85]. Simulation results also show little difference between WG mode and Leaky modes in lower order modes for both hexagonal and cylindrical structures. As with the LMR, the resonant WG modes have been used as the basis for a precise theoretical explanation of the enhanced optical behavior of hexagonal NWs, such as enhanced light absorption [81, 93-96] and emission [78, 97-99]. Furthermore, these numerical solutions have lead to reproduction of experimental resonance spectra, e.g., polarization-resolved micro-photoluminescence (µ-PL) and cathodeluminescence (CL) spectroscopy. 
\subsection{Fabry-Perot Resonant Mode}
The above analysis and results apply to long structures, hence, provide two-dimensional radial modes, independent of the NW axis. However, light confinement has strong axial dependence, necessitating three-dimensional analysis of the cavity modes. FDTD simulation in 3D are used to identify the axial dependence of resonant modes in these nano-structures, revealing modes which are volumetric in nature.
Fabry-Perot (FP) modes have been analyzed for sub-microcavity, or nano-cavity, NWs with cylindrical or hexagonal structures, specifically in order to determine the axial dependence of the resonance modes [100]. At least two mirrors are needed to construct the reflection structure inside the cavity, whether they are the top and bottom ends, i.e., the air and substrate interfaces with the nanowire, or any of the two opposite facets along the nanowire axis. For subwavelength structures, the longitudinal WG modes have high scattering losses due to diffraction, and axial FP waveguide modes will dominate [90]. However, due to small difference of the refractive index between the substrate and the nanowire dielectric, the existence of the FP mode will only be valid if the nanowire has relatively large radii, e.g., larger than 200 nm [101]. Under these conditions, besides the top and bottom ends, the lateral facets of nanowire can also be treated as two parallel slabs, and with the dielectric in between, it can support the FP mode with mode spacing inversely related to the nanowire length. An application of this analysis is in the design of NW lasers, since the optical cavity modes are observed at threshold for lasing, and have been investigated for both optical and electrical pumped cases [102, 103]. As a results the FP resonance mode based nanoscale lasers are not only capable of covering a wide spectral regions, but can also can be integrated as single or multi-color laser source arrays in silicon based photonic integrated circuit or microelectronic devices [102,103]. However, the FP modes supported by the nano-cavity structure have relatively small quality factor due to the small difference of the refractive indices of the substrate and the NWs. In order to address this issue,  Bragg gratings can be produced at the NW ends, alternatively, NWs can be placed on metal substrates in order to increase the FP resonance peak intensity by more than one order of magnitude compared to those on Si substrates [104].
\subsection{Helical Resonance Modes}
Nano needles of III-V material grown on heterogeneous substrates are optoelectronic devices which have shown interesting optical behavior, including lasing, at room temperature [105]. Figure 3 (A) shows SEM image of a nano-laser grown on silicon substrate that has subwavelength dimensions on all sides. Analysis of light propagation introduced by shows that unlike the traditional WG mode that lack vertical structure, there is net propagation in axial direction in these structures which leads to  volumetric resonant modes which are termed helical mode resonances [105]. The schematic Fig. 3(B) suggests a helical ray path with nearly total internal reflection at the nanopillar-silicon interface due to the glancing angle of incidence from the hexagonal facets of the nano-laser shown in Fig. 3(A). As such, the faceted shape of the structure affects the optical cavity properties. FDTD-simulated field profile shows a hexagonal WG-like mode pattern  in the transverseplane as in Fig. 3 (C), which arises from strong azimuthal components of helical modes. Figure 3 (D) shows first-order  and higher-order standing waves’ axial variation. The radial mode number (first number, m) describes the transverse field pattern for WG modes, and the axial mode number (second number, n) describes the axial standing wave as is the case for Fabry-Perot resonances. It is seen that light or optical field can be well confined in the nanostructure even with low index contrast at the dielectric interface thus producing the nano-resonators needed for lasing. Although the quality (Q) factors of such nanostructure are usually not large, these helically propagating cavity modes, provide an optical feedback mechanism without the sophisticated mirror structures of the vertical cavity surface emitting lasers (VCSEL’s). Additionally, since the nanowires are heteroepitaxially grown on different substrates, they enable heterogeneous integration of photonic emitters and silicon based computational circuitry.  Whereas traditional FP modes are inhibited by the interface between semiconductor nanostructure and the silicon substrate, such unique optical structures have been proposed as an avenue for engineering and integrating on-chip nanophotonic devices. 

\section{Volumetric Modes} \label{sed}
The diameter of the nanostructures which can support the helical resonance modes is near the Rayleigh limit, around the boundary of the validity of ray-optics. FDTD analysis can be applied to deeper subwavelength structure in order to identify the cavity modes which are by nature volumetric, i.e., axially dependent.  Figure 4 shows simulation results for various diameters of hexagonal structure of 1 m length.  Incident radiation with 532 nm wavelength is nearly parallel to the wire axis and different modes are displayed for different radii. Top row shows radial spatial dependence at the middle of the wire axis, and the bottom row shows the axial dependence. Top row results are similar to Figures 1-2, and the bottom row shows that the light can be confined in volumetric resonance mode in both transverse plane and longitudinal plane even with sub-wavelength diameter of these hexagonal NWs. Unlike helical modes, the explanation of resonance  need not rely on an intuitive ray-optics description based on the grazing angle of incident light, but shows similar results in how the deep subwavelength structures can confine the light and produce a resonant cavity without having sophisticated mirrors at the end facets. In this respect nano-cavities of as-grown nanowires outperform microcavities of VCSELs.
\subsection{Nanocavity Geometry Dipendence}

\subsection{Light Engineering of Nanocavities}
Dependence of the resonant modes on the cavity geometry offers an important degree of freedom to engineer a cavity for particular optical properties. Figure 6 shows the dependence of three volumetric TM resonant modes’ excitation wavelengths with radius. In this spectral range, only lower TM modes can be excited with smaller radii, e.g., r = 40 nm and 60 nm, however, as the radius increases, higher order modes can be excited, and the optical power corresponding to the lower order modes will be reduced. We observe redshift of these volumetric TM modes with increasing NW radius. Also, the wavelength variation of TM1n mode is much larger compared to TM2n and TM3n modes. These observations demonstrate the feasibility to engineer the volumetric mode at certain wavelength, i.e., allow us to optimize absorption or emission at a desired frequency or certain incident optical power  by controlling the radius and/or length of a NW thus providing the ability to engineer the absorption spectrum in order to match desired properties.

iThe dependence of the resonant modes on NW radius also suggests the interesting possibility of having tapered structures which can support more than one resonant mode, thus be able to optimize the spectrum of interest.  The metalorganic vapor phase epitaxy (MOVPE) or vapor liquid solid (VLS) growth methods are readily capable of forming nanowires with tapered sidewalls. The resultant cavity, however, does not support the superposition of the modes present in cylindrical structures of the same diameter; in fact tapered sidewalls have been identified as the primary loss mechanism for these sub-wavelength cavities.  The effect of  tapering has been studied for  nanopillars that were grown on a silicon substrate with average 5° angles between opposite sidewalls; vertical field profiles for ,  and  modes are shown in Fig. 7 [105]. The modes are primarily confined at the base, and become less resonant as they propagate upwards with decreasing of the radius at top. Higher-order axial modes generally have lower quality factor. Physically, the stronger Fabry-Perot characteristic of higher-order axial modes means that their effective longitudinal wave-vector components become stronger, causing larger penetration and loss into the substrate. Nevertheless, from a different perspective, multi-mode resonances can be achieved within certain wavelength range by controlling the tapering angle in order to form small varying radius along the nanostructure axial direction. One can also red- or blue-shift the resonance peaks, since these volumetric resonance modes are dependent on transverse dimensions. Thus, intentioned tapering offers an alternative way to engineering the multi-mode resonances and finer tunability of these resonance peaks.

%\subsubsection{\civ-dependent Mean} \label{CIV SED}



