\chapter{Conclusions and Future Research} \label{conclusions}

As mentioned previously, communication of information, together with storage and computation form a “grand challenge” of the information age [2, 7]. Recently, the analysis of big data has become the engine for societal, financial, scientific, and technological endeavors. This demands an infrastructure that is capable of fast and reliable high volume data processing. Traditionally, this requirement was fulfilled by silicon technology. However, silicon-based technology has its own limitations, such as speed limit and heat dissipation problem. In order to process high volume data, we need data computation, storage and communication work as three fundamental functions of a computation cell. A monolithic nano-system may be envisioned which incorporates NWs as waveguides, detectors, photovoltaic cells, antennas, modulators, (photo)capacitors, LEDs, and lasers, .These components may be incorporated in circuit layers, such as network on chip. Different layers can communicate using NW through-silicon vias (TSVs). Similar low-power/high-performance advantages can be realized through achievement of high interconnect densities on the 2.5D Though-Si-Interposer (TSI) as reported in [114]. 
In conclusion, optical properties of nano-cavities were reviewed here emphasizing the analysis of resonant optical modes which depend both radially and axially on the geometries of the nanowires. This shows how such sub-wavelength structures can form optical cavities as-grown, without needing sophisticated facet mirrors. In addition, we show how the fortuitous overlap of the reduced dimensional electronic wave functions and the photonic modes is responsible for the extraordinary optoelectronic properties of core-shell nanowires. Such nano-structures have been developed on heterogeneous substrates, particularly silicon, and as such becoming an important component in the next generation of photonic integrated circuits which are particularly useful in meeting the grand challenge of low energy and fast speed computation.
\section{Contributions of this dissertation}
\section{Outline of the future work}

%Insert amazing opening paragraph here.  An overview of the goals of this thesis.
%General idea: Take a group of observables, look for correlations, relate to physics.

Low dimensional electron gases exsit at the heterointerfaces of core-shell nanowires (CSNWs). For example, the GaAs/AlGaAs CSNWs typically form a hexagonal structure in which six (6) pillars of 1D charge at the vortices, and six (6) sheets of 2D charge at facets  are formed [1]. At the same time, nanowires (NW) have also been shown to be capable of confining light in their sub-wavelength nano-structure, supporting photonic modes, and producing resonant cavities without the need for polished end facets. We have previously shown how the electronic wave functions that are thus formed affect the optical transition rates, resulting in orders of magnitude  enhancement in absorption and emission of light [ref]. Here we report on the plasmonic effects of the confined charge on the optical properties of CSNWs We report on finite difference time domain (FDTD) simulations with the aim of identifying the surface plasmon resonance modes which affect light confinement in hexagonal CSNWs, and help form a  high quality factor resonant cavity. This is done by comparing regular CSNW, with a) wires covered with metal which produces surface plasmon-polaritons (SPP’); b) NWs covered with metal that is sandwiched between the core and the outer, shell; and c) two-dimensional electron gas (2DEG)  which  embedded at the heterointerace of CSNWs. Results show that the 2DEG behaves similarly to an embedded metallic surface, allowing for highly localized light confinement in these wires without the need for vertical structures such as Bragg mirrors commonly used in vertical cavity surface emitting lasers (VCSEL’s). Besides affecting the cavity, the 2DEG enhances  the transition rates due to the plasmon-electron interaction, facilitating not only photonic stimulated emission and lasing, but also  surface plasmon amplification by stimulated emission of radiation [2].
    We model the dielectric function of the two dimensional electron gas using the Drude model for dispersive media, and extract its relevant parameters from [3]. The complex conductivity of the 2DEG is derived using the relaxation time approximation, effective reduced mass of electrons, and the density of the carriers in the gas. By substitution the complex conductivity in Drude model, we can model the 2DEG, with given plasma frequency, damping coefficient, and the oscillator strength using FDTD simulator.
    The two dimensional plasma frequency is calculated as [3]:
    n which,  is the background dielectric constant and m* is the effective mass of the electron. It is important to note that, as shown in (1), the plasma frequency of the 2DEG can be tuned with changing the carrier concentration. This tubnability distinguishes the 2DEG from other plasmonic material such as metals. The complex conductivity of the electron gas is derived as [4]:

%In Chapter~\ref{data} we brought together data from the mid-IR through the UV for the purpose of creating multi-wavelength SEDs for the SDSS DR7 quasars catalog.  This involved cross-matching mid-IR data from {\em Spitzer} and {\em WISE}, near-IR data from 2MASS and UKIDSS, and UV data from {\em GALEX}.  From this cross matched data set we created several subsets used throughout our studies.  We began with a (observationally) non-reddened data set used to explore trends in the SEDs based on various observed properties. To study the dust reddening properties of our quasars we limited our data to quasars uniformly selected by the SDSS quasar detection pipeline.  This data set was further split into quasar with BALs and quasars without BALs.  Finally, when exploring our SEDs as function of $\mbh$ we further limited this sample to the non-BAL quasars since BALs can make the estimated values for $\mbh$ invalid.

